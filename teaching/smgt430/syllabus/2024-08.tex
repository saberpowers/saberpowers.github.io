\documentclass[11pt]{article}
\usepackage{fullpage}
%% color for the links 
\usepackage[usenames,dvipsnames]{color}
\usepackage{longtable, graphicx}
\usepackage{ebgaramond}		% for boldface small caps
\usepackage{changepage}		% for indented paragraph blocks
% We will generate all images so they have a width \maxwidth. This means
% that they will get their normal width if they fit onto the page, but
% are scaled down if they would overflow the margins.
\makeatletter
\def\maxwidth{\ifdim\Gin@nat@width>\linewidth\linewidth
\else\Gin@nat@width\fi}
\makeatother
\let\Oldincludegraphics\includegraphics
% \renewcommand{\includegraphics}[1]{\Oldincludegraphics[width=\maxwidth]{#1}}
%% hyperlinks
\usepackage[
	colorlinks=true,
	urlcolor=MidnightBlue,
	plainpages=false,
  	]{hyperref}% color for the links 
\setlength{\parindent}{0pt}
\setlength{\parskip}{6pt plus 2pt minus 1pt}
\setlength{\emergencystretch}{3em}  % prevent overfull lines
\setcounter{secnumdepth}{0}

\begin{document}

\begin{minipage}[c]{0.4\linewidth}
  \includegraphics[height = 2cm]{../../images/rice_smgt.png}
\end{minipage}
\begin{minipage}[c]{0.6\linewidth}
  \raggedleft
  {\bf Course:} SMGT $430/530$ -- Introduction to Sport Analytics\\
  {\bf Term:} Fall $2024$\\
  {\bf Class:} $2$:$30$ p.m. -- $3$:$45$ p.m. Tue/Thu
\end{minipage}

\hrule

~\\
\textbf{\textsc{Instructor Contact Information}}
\begin{adjustwidth}{1cm}{0cm}
	{\bf Instructor:} Scott Powers\\
  {\bf Office:} Kraft $246$\\
  {\bf Email:} scott.powers@rice.edu\\
  {\bf Office Hours:} calendly.com/saberpowers
\end{adjustwidth}

~\\
\textbf{\textsc{Required Texts and Materials}}
\begin{adjustwidth}{1cm}{0cm}
	{\bf Class Website:} canvas.rice.edu\\
	{\bf Required Text:} All readings will be made available on Canvas.\\
	{\bf Required Material:} Bring your laptop to class every day, but please leave your laptop packed away when we are not working on coding exercises.
\end{adjustwidth}

~\\
\textbf{\textsc{Topics}}
\begin{adjustwidth}{1cm}{0cm}
  The general flow is that we will alternate between (a) discussing math/models using a whiteboard and (b) implementing these concepts in code. The course is divided into $4$ units, each approximately $4$ weeks:
  \begin{itemize}
    \item Unit $1$: Estimating Team and Player Strength\\
    Pythagorean Formula, Bradley-Terry Model, Plus-Minus Models
    \item Unit $2$: Reducing Noise in Player Evaluation\\
    Regression to the Mean, Regularized Regression, Regularized Adjsted Plus-Minus
    \item Unit $3$: Applications of Markov Chains in Sports\\
    Markov Chains, Win Probability, Credit Attribution, In-Game Strategy, Markov Decision Processes
    \item Unit $4$: Practicum\\
    Guest Speakers, Student Projects
  \end{itemize}
\end{adjustwidth}


~\\
\textbf{\textsc{Course Objectives and Learning Outcomes}}
\begin{adjustwidth}{1cm}{0cm}
  After successfully completing this course, you will be able to:
  \begin{itemize}
    \item Use R to apply theoretical statistical models to real-world sports data, and interpret the results.
    \item Calculate estimates of team and player strengths, adjusting for sample size and strength of schedule.
    \item Estimate in-game win probability models based on game state (e.g. score, possession, time remaining).
    \item Identify an interesting research problem in sports, solve it with data, and present your solution.
  \end{itemize}
\end{adjustwidth}

\newpage
\textbf{\textsc{How Your Grade Is Determined}}
\begin{adjustwidth}{1cm}{0cm}
	Sport analytics is a demanding and highly competitive field. This class aims to provide the tough but honest feedback that you will need to succeed in it.
  \begin{center}
    \begin{tabular}{rclcrclcrcl}
                      &               &     & & $[93\%,\,\infty)$ & $\rightarrow$ & A & & $[90\%, 93\%)$ & $\rightarrow$	& A--\\
      $[87\%, 90\%)$  & $\rightarrow$	& B+  & & $[83\%, 87\%)$  & $\rightarrow$ & B & & $[80\%, 83\%)$ & $\rightarrow$	& B--\\
      $[77\%, 80\%)$  & $\rightarrow$	& C+  & & $[73\%, 77\%)$  & $\rightarrow$ & C & & $[70\%, 73\%)$ & $\rightarrow$	& C--\\
      $[67\%, 70\%)$  & $\rightarrow$	& D+  & & $[63\%, 67\%)$  & $\rightarrow$ & D & & $[60\%, 63\%)$ & $\rightarrow$  & D--\\
                      &               &     & & $[ 0\%, 60\%)$  & $\rightarrow$ & F\\
    \end{tabular}
  \end{center}
  ~\\
	\textbf{Assignments.} ($40\%$) There are $4$ individual assignments, each worth $10\%$. You are encouraged to help each other, but your submission must reflect work you have done yourself. Assignments will be graded anonymously, so please remove any personally identifying information from your submission. The expected completion time for each assignment is $10$ hours.

  \textbf{Project.} ($40\%$) This is a project-based course, and the project you build in stages over the course of the semester will determine a large chunk of your grade. You are encouraged to work with a partner on the project (no more than $2$ students per team), but you may work individually if you wish.
  \begin{itemize}
    \item Project \#$0$ ($0\%$): Registration                   \hfill expected hours: $1$
    \item Project \#$1$ ($5\%$): Proposal                       \hfill expected hours: $5$
    \item Project \#$2$ ($10\%$): Abstract                      \hfill expected hours: $10$
    \item Project \#$3$ ($5\%$): Presentation                   \hfill expected hours: $5$
    \item Project \#$4$ ($20\%$): Paper                         \hfill expected hours: $20$
  \end{itemize}

  \textbf{Midterm.} ($10\%$) There is a midterm exam that carries the same weight as a homework assignment.

  \textbf{Attendance.} ($10\%$) Showing up is a requirement for most jobs in sport analytics. In this class, you are expected to show up and engage in discussions, which are part of the learning experience. We will keep attendance and collect feedback using exit tickets at the end of each class.

  \textbf{Late Work.} You may submit late work subject to a $10\%$ score deduction for each late day.

  \textbf{Absence Policy.} If you notify me of your absence before the beginning of class, you will receive an opportunity to make up your absence.
\end{adjustwidth}

\newpage
\textbf{\textsc{Rice Honor Code}}
\begin{adjustwidth}{1cm}{0cm}
  In this course, all students will be held to the standards of the Rice Honor Code, a code that you pledged to honor when you matriculated at this institution. If you are unfamiliar with the details of this code and how it is administered, you should consult the Honor System Handbook at \url{http://honor.rice.edu/honor-system-handbook/}. This handbook outlines the University's expectations for the integrity of your academic work, the procedures for resolving alleged violations of those expectations, and the rights and responsibilities of students and faculty members throughout the process.\\
  ~\\
  \textbf{AI Policy.} You are allowed (and even encouraged) to use AI as a tool for developing R code (attribution in this case is not necessary). Please do not use AI to generate any part of your writeups.
\end{adjustwidth}

~\\
\textbf{\textsc{Disability Resource Center}}
\begin{adjustwidth}{1cm}{0cm}
  If you have a documented disability or other condition that may affect academic performance you should: $1$) make sure this documentation is on file with the Disability Resource Center (Allen Center, Room $111$ / \href{mailto:adarice@rice.edu}{adarice@rice.edu} / x$5841$) to determine the accommodations you need; and $2$) talk with me to discuss your accommodation needs.
\end{adjustwidth}

~\\
\textbf{\textsc{Mental Health Policy}}
\begin{adjustwidth}{1cm}{0cm}
	The wellbeing and mental health of students is important; if you are having trouble completing your coursework, please reach out to the Wellbeing and Counseling Center. Rice University provides cost-free mental health services through the Wellbeing and Counseling Center to help you manage personal challenges that threaten your personal or academic well-being. If you believe you are experiencing unusual amounts of stress, sadness, or anxiety, the Student Wellbeing Office or the Rice Counseling Center may be able to assist you. The Wellbeing and Counseling Center is located in the Gibbs Wellness Center and can be reached at 713-348-3311 (available 24/7).
\end{adjustwidth}

~\\
\textbf{\textsc{Title IX Responsible Employee Notification}}
\begin{adjustwidth}{1cm}{0cm}
  At Rice University, unlawful discrimination in any form, including sexual misconduct, is prohibited under Rice Policy on Harassment and Sexual Harassment (Policy 830) and the Student Code of Conduct.

  Please be aware that all employees of Rice University are ``mandatory reporters'', which means that if you tell me about a situation involving sexual harassment, sexual assault, dating violence, domestic violence, or stalking, I must share that information with the Title IX Coordinator.

  Although I have to make that notification, you will control how your case will be handled, including whether or not you wish to pursue a formal complaint. Our goal is to make sure you are aware of the range of options available to you and have access to the resources you need.

  To report sexual harassment, please contact the Title IX Coordinator at titleix@rice.edu. To explore supportive measures and other resources that are available to you, please visit the Office if Interpersonal Misconduct Prevention and Support at safe.rice.edu.
\end{adjustwidth}

~\\
\textbf{This syllabus is only a guide for the course and is subject to change with advance notice.}

\end{document}
