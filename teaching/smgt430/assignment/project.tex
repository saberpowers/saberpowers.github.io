

\documentclass{article}

\usepackage[margin=1in]{geometry}

\begin{document}

  \section*{\sc SMGT 430 Project Overview}

    The class project is a research paper and presentation on any topic of your choosing. Try to pick an interesting research question in sport analytics, and try to answer it using the tools we've been learning in class (and any tools you bring from outside of class). You may work individually or with a partner.\\
  
  \hrule
  \section*{\sc Project \#0: Registration}

    \subsection*{\sc Why are you being asked to do this?}

      This assignment is a forcing function to get you started on thinking about your project.
    
    \subsection*{\sc What (exactly) are you being asked to do?}

    Tell me who your partner will be (or that you'll be working individually), and list five potential research questions. The best way to come up with one good idea is to force yourself to think up a bunch and then filter out the bad ideas. For each research question, please provide 1-3 sentences explaining the idea.

      \subsubsection*{\sc Submission Requirements}

        Please submit your registration using the text entry box on the assignment page in Canvas. If you are working with a partner, only one of you needs to submit this assignment.

    \subsection*{\sc How will you be graded?}

      This assignment will not count toward your semester grade.\\

  \hrule
  \section*{\sc Project \#1: Proposal}

    \subsection*{\sc Why are you being asked to do this?}

      This assignment is a forcing function to get you focused on a specific research question for your project.

    \subsection*{\sc What (exactly) are you being asked to do?}

      Write no more than 250 words addressing the following questions:
      \begin{itemize}
        \item What research question will you attempt to answer for your class project?
        \item What decisions could a sports team improve using the answer to this research question?
        \item What data will you need, and how will you get it?
        \item What methods will you use to answer the research question?
        \item What is the closest anyone has come to answering this research question in the public domain?
      \end{itemize}

      \subsubsection*{\sc Submission Requirements}

        Please submit your proposal using the text entry box on the assignment page in Canvas. If you are working with a partner, only one of you needs to submit this assignment.

    \subsection*{\sc How will you be graded?}

      This assignment will be graded for completion, meaning that as long as you complete the assignment, your grade will be 100\%. You will get feedback on the impact and feasibility of your proposal.\\

  \newpage
  \section*{\sc Project \#2: Abstract}

    \subsection*{\sc Why are you being asked to do this?}

      When presenting research at a sport analytics conference, the first step is often submitting an abstract that describes your work. Generally, the decision of whether or not you are invited to present at the conference is based on this abstract. In addition to being great practice for a future research conference presentation, this abstract also serves as a checkpoint so that you don't fall behind on your project paper.

    \subsection*{\sc What (exactly) are you being asked to do?}

      A strong abstract describes the data and methodology (in greater detail than your submission for Project \#1: Proposal) and demonstrates some initial results. For this abstract, describe some initial results to demonstrate that you are on track. You don't need to have your complete results for the project, but you want to show that you have gotten your hands dirty with the data.

      \subsubsection*{\sc Submission Requirements}

        \begin{itemize}
          \item A PDF report (max 2 pages) summarizing your findings, including at minimum the following:
          \begin{itemize}
            \item a clear statement of your research question and an explanation of why it is important
            \item a description of the methodology you used to produce your initial results.
            \item a summary of your initial results
            \item a data visualization that tells an interesting story about your initial results
            \item a description of the work remaining to complete your results
          \end{itemize}
          \item An R script that contains all of the code you used to perform the research
        \end{itemize}

    \subsection*{\sc How will you be graded?}

      You will get feedback on your work product based on several criteria. Within each of those criteria, the feedback will be: Missing (0\%), Needs Improvement (70\%), Good (85\%) or Exceeds Expectations (100\%). Your grade on the assignment will be the average of the grades across criteria. The criteria are:
      \begin{enumerate}
        \item {\bf Progress.} Have you made substantial progress toward answering your research question?
        \item {\bf Impact.} Have you addressed a research question that could impact a decision in sport management or that could change how fans think about the sport? How useful is the answer to your research question?
        \item {\bf Methods.} How appropriate are your methods for answering your research question? We don't need sophistication for the sake of sophistication, but we do need to use the right tool for the job. The best research questions are high-impact questions that can be answered with simple methodology.
        \item {\bf Data visualization.} Have you grabbed the attention of the abstract reviewer?
        \item {\bf Remaining work.} Were you specific in your descriptions of the remaining steps? Have you made it clear how you will finish answering your research question?
      \end{enumerate}
  
  \hrule
  \section*{\sc Project \#3: Presentation}

    \subsection*{\sc Why are you being asked to do this?}

      When sharing research in sport analytics, the presentation you make at a conference is often the most impactful way to get your work more exposure. This assignment is an opportunity for you to practice your oral presentation skills and to learn about the projects your classmates have been working on.
    
    \subsection*{\sc What (exactly) are you being asked to do?}

      During the final week of the semester, each group will present their project to the class. Your research should be complete at this point although you still have more time to finish preparing your paper. The deadline for slide submission is the same for everyone, and we will randomly determine the presentation order at the beginning of the presentation week. You will have 7 minutes to present, followed by 3 minutes of Q\&A. Generally, a good rule of thumb is that you want to have fewer slides than minutes of presentation, so I would recommend preparing 5-6 slides to explain your project to your classmates.

    \subsection*{\sc How will you be graded?}

      This assignment will be graded for completion, meaning that as long as you complete the assignment, your grade will be 100\%.\\

  \hrule
  \section*{\sc Project \#4: Paper}

    \subsection*{\sc What (exactly) are you being asked to do?}

      Write a paper detailing your research project. The exact format of the paper is up to you. A longer paper does NOT necessarily correspond with a better grade. Focus on communicating efficiently the details of what you did, not filling the paper with fluff to make it longer. You want the paper to have enough detail that a reader could reproduce your work, but you do not need to teach the reader how to do things in R. A research paper will often have sections like: Introduction, Background, Data, Methods, Results, Discussion, etc., but it is NOT necessary for you to follow this format. It's up to you how many figures/tables you want to include in your paper.
  
      \subsubsection*{\sc Submission Requirements}

        \begin{itemize}
          \item You paper as a PDF (max 8 pages)
          \item An R script that contains all of the code you used to perform the research
        \end{itemize}
   
    \subsection*{\sc How will you be graded?}
 
      You will get feedback on your work product based on several criteria. Within each of those criteria, the feedback will be: Missing (0\%), Needs Improvement (70\%), Good (85\%) or Exceeds Expectations (100\%). Your grade on the assignment will be the average of the grades across criteria. The criteria are:
      \begin{enumerate}
        \item {\bf Impact.} Have you addressed a research question that could impact a decision in sport management or that could change how fans think about the sport? How useful is the answer to your research question?
        \item {\bf Methods.} How appropriate are your methods for answering your research question? We don't need sophistication for the sake of sophistication, but we do need to use the right tool for the job. The best research questions are high-impact questions that can be answered with simple methodology.
        \item {\bf Critical thinking.} Did you draw appropriate conclusions from your results?
        \item {\bf Written communication.} Did you write clearly and concisely? Did you organize your key ideas with the evidence supporting them? Did you format your report in a way that makes it easy to read?
      \end{enumerate}

\end{document}