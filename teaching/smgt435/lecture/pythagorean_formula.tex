
\documentclass{article}

\usepackage{amsfonts}
\usepackage{amsmath}
\usepackage{fancyhdr}
\usepackage{framed}
\usepackage[margin=1in]{geometry}

\pagestyle{fancy}
\fancyfoot[RE,RO]{
  \sc \footnotesize
  SMGT 435: Baseball Analytics\\
  \copyright~2024 Scott Powers}


\begin{document}

  \begin{framed}
    {\bf Caution:} These lecture notes are under construction. You may find parts that are incomplete.
  \end{framed}

  \section{\sc Pythagorean Formula}

    \subsection{\sc When do we switch to preferring actual winning percentage?}

      $n_i$ is the number of games played by team $i$\\
      $X_i$ is the Pythag W\% of team $i$\\
      $Y_i$ is the actual W\% of team $i$\\
      $Z_i$ is the residual W\% of team $i$ ($Y_i = X_i + Z_i$)

      \subsubsection*{\sc Intuition}

      \begin{align*}
        \mbox{Actual W\%} &= \mbox{Pythag W\%} + \mbox{Residual}\\
        \mbox{Outcome} &= \mbox{Skill} + \mbox{Luck}\\
        Y_i &= X_i + Z_i
      \end{align*}

      \subsubsection*{\sc Model}

      \begin{align*}
        X_i &\sim \mbox{ind. Normal}(\mu_i,\, \sigma^2_X / n_i)\\
        \mu_i &\sim \mbox{i.i.d. Normal}(\mu_0,\, \sigma^2_\mu)
      \end{align*}

      Option \#1 ($Z_i$ is all luck):
      \begin{equation*}
        Z_i \sim \mbox{i.i.d. Normal}(0,\, \sigma^2_Z / n_i)
      \end{equation*}

      Option \#2 ($Z_i$ is not purely luck):
      \begin{align*}
        Z_i &\sim \mbox{ind. Normal}(\eta_i,\, \sigma^2_Z / n_i)\\
        \eta_i &\sim \mbox{i.i.d. Normal}(0,\, \sigma^2_\eta)
      \end{align*}
    
      For $Z_i$, the signal variance is $\sigma^2_\eta$, and the noise variance is $\sigma^2_Z / n$. The total variance is $\sigma^2_\eta + \sigma^2_Z / n$. When $n = \sigma^2_Z / \sigma^2_\eta$, the variance in $Z_i$ is half signal, half noise.

      A common measurement of interest for evaluating metrics in baseball is the {\it split-half correlation}. A high split-half correlation close to one tells you that a metric is stable and reliable. A lower split-half correlation close to zero tells you that a metric is noisy and unreliable. We can imagine splitting the season into two halves and calculating the residual winning percentages $Z_i^1$ and $Z_i^2$ in the first half and second half respectively. Assuming equal sample sizes $n_i = n_i^1 = n_i^2$,
      \begin{align*}
        \mbox{Corr}(Z_i^1, Z_i^2) &= 
        \frac{\mbox{Cov}(Z_i^1, Z_i^2)}{\sqrt{\mbox{Var}(Z_i^1) \mbox{Var}(Z_i^2)}}
        = \frac{\mbox{Cov}(\eta_i, \eta_i)}{\sqrt{(\sigma^2_\eta + \sigma^2_Z / n^1_i)(\sigma^2_\eta + \sigma^2_Z / n^2_i)}}\\
        & = \frac{\mbox{Var}(\eta_i)}{\sqrt{(\sigma^2_\eta + \sigma^2_Z / n_i)(\sigma^2_\eta + \sigma^2_Z / n_i)}}
        = \frac{\sigma^2_\eta}{\sigma^2_\eta + \sigma^2_Z / n_i}.
      \end{align*}
      Again we see the significance of $n = \sigma^2_Z / \sigma^2_\eta$ because this sample size makes the split-half correlation 0.5.

      Lastly, if our goal is to use observed results to predict future results, then $\mu_i + \eta_i$ is what we want to estimate. We can compare how well $X_i$ and $Y_i$ achieve this goal.
      \begin{align*}
        E[(X_i - (\mu_i + \eta_i))^2] &= E[((X_i - \mu_i) + \eta_i)^2]\\
        &= E[(X_i - \mu_i)^2] + 2E[(X_i - \mu_i) \cdot \eta_i] + E[\eta_i^2]\\
        &= \sigma^2_X / n + 0 + \sigma^2_\eta
        = \sigma^2_X / n + \sigma^2_\eta.
      \end{align*}
      By contrast,
      \begin{align*}
        E[(Y_i - (\mu_i + \eta_i))^2] &= E[((X_i - \mu_i) + (Z_i - \eta_i))^2]\\
        &= E[(X_i - \mu_i)^2] + 2E[(X_i - \mu_i) \cdot (Z_i - \eta_i)] + E[(Z_i - \eta_i)^2]\\
        &= \sigma^2_X / n + 0 + \sigma^2_Z / n
        = \sigma^2_X / n + \sigma^2_Z / n.
      \end{align*}

      We see that $E[(Y_i - (\mu_i + \eta_i))^2] < E[(X_i - (\mu_i + \eta_i))^2]$ when $n > \sigma^2_Z / \sigma^2_\eta$. In other words, actual record ($Y_i$) becomes a stronger prediction of future record than Pythagorean record ($X_i$) when the number of games observed is at least $\sigma^2_Z / \sigma^2_\eta$.

    \subsection{\sc Discussion Questions}

      \begin{enumerate}
        \item Why is the Pythagorean formula important?
        \item What properties of a sport would cause the Pythagorean exponent to be bigger or smaller?
        \item What are some reasons a team might truly be better than their Pythagorean record suggets?
        \item If a team plays 10 games, what do you think will be more predictive of their future record: Pythag W\% or actual W\%? If the team plays 10 million games? At how many games do you switch from preferring Pythag W\% to actual W\%?
        \item What properties of a sport would cause past Pythagorean W\% to be more or less predictive of future actual W\% (relative to past actual W\%)?
        \item The Pythagorean formula is elegant but simple. How might you develop a more sophisticated formula to predict W\% from runs scored and runs allowed?
        \item What MLB teams have most underperformed or overperformed their Pythagorean records this season? What are the implications of this observation?
      \end{enumerate}

\end{document}
